\documentclass[conference]{IEEEtran}
\IEEEoverridecommandlockouts

\usepackage{cite}
\usepackage{amsmath,amssymb,amsfonts}
\usepackage{algorithmic}
\usepackage{graphicx}

\graphicspath{{./images/}}

\newcommand{\columnFigure}[2]{
    \begin{figure}[!h]
        \centering
        \includegraphics[width = 7.5cm]{#1}
        \caption{#2}
        \label{fig:#1}
    \end{figure}
}

\newcommand{\fullWidthFigure}[2]{
    \begin{figure*}[!h]
        \centering
        \includegraphics[width = 16cm]{#1}
        \caption{#2}
        \label{fig:#1}
    \end{figure*}
}

\begin{document}

\title{
    {\large Recent Advances in Recurrent Neural Networks -- Winter Semester 2019/2020}\\
    Cortical microcircuits\\as gated-recurrent neural networks
}

\author{
    \IEEEauthorblockN{Paweł A. Pierzchlewicz}
    \IEEEauthorblockA{Graduate Training Center for Neuroscience\\University of Tübingen, Tübingen, DE}
}

\maketitle

\begin{abstract}
A summary paper of "\textit{Cortical microcircuits as gated-recurrent neural networks}" by Rui Ponte Costa et al. Cortical circuits show recurrent architectures across different brain areas. This suggest existance of computational principles, which still remain elusive. The authors propose a modified method of long short-term memory networks, which are meant to be more biologically plausible. They introduce a network model, which uses inhibitory cells as subtractive gates (subLSTMs). This allowed them to propose a mapping of subLSTMs onto known cortical excitatory-inhibitory cortical microcircuits. Empirical evaluation was performed on sequential image classification and language modelling. The results suggest that cortical circuits could be optimised to solve complex contextual problems. Their work provides a step towaeds unifying deep learning with their biological counterparts.
\end{abstract}

\begin{IEEEkeywords}
Cortical Microcircuits, Recurrent Neural Networks, Computational Neuroscience
\end{IEEEkeywords}

\section{Introduction}
This is a summary of "\textit{Cortical microcircuits as gated-recurrent neural networks}" by Rui Ponte Costa et al \cite{PonteCosta2017}.
The past decades have unveiled complex but stereotypical structures within the neocortex. The most prevalent feature of the cortical networks, being their laminar organisation and a high degree of recurrence \cite{douglas_koch_mahowald_martin_suarez_1995, jiang_shen_cadwell_berens_sinz_ecker_patel_tolias_2015}. Thus Recurrent Neural Networks (RNNs) and Long-Short Term Memory units (LSTMs) seem like an appealing idea for modelling cortical structures. There is however a major difference between the two architectures. LSTMs have a multiplicative factor for gating while biological networks use inhibitory neurons, which have a subtractive nature. There have been suggestions of LSTMs as models of working memory and different brain areas \cite{krueger_dayan_2009}, but without clear interpretation of the individual components of LSTMs and specific mapping to known circuits.
By transferring the functionality of LSTMs into a more biologically plausible network the work is meant to provide a testable hypotheses for experiments on the functionality of entire cortical microcircuits.
\section{Biological Inspiration}
The work was inspired by the columnar architecture of the cortical circuits presented on \figurename{\ref{fig:biomodel}}. It partially describes the flow of sensory information from the thalamus to the neocortex. The sensory input is propagated to layer 2/3 pyramidal cells. The output of layer 2/3 is projected to layer-5 pyramidal cells, where recurrent units may form. The input into the layer-5 pyramidal neurons is gated by local inhibitory connections. As a result it may retain a given input for some time. The timescale of this is dependant on the decay time constant. The output of layer-5 is then projected to other units of the network, with the output controlled by another inhibitory neuron.

\columnFigure{biomodel}{Biological architecture of the neocortex, which resembles LSTM features. \textbf{1.} The thalamus projects the sensory input to layer 2/3 pyramidal cells. \textbf{2.} The output of layer 2/3 pyramidal cells is projected to layer-5 pyramidal cells. These may form recurrent units analogous to the memory cell of LSTMs. Here local interneurons control gate the input into the memory cell (LSTM input gates \textit{i}). \textbf{3.} The recurrences allow for retaining a given input for some time dependant on the decay time constant \textit{f}. \textbf{4.} Finally the output is sent to other units, where the output is gated by another local inhibitory cell (LSTM output gate \textit{o}).}

\subsection{LSTMs vs. Biological Circuits}
When compared to LSTMs significant resemblance can be found. The central element of LSTMs, the \textit{memory cell} is hypothesised to be implemented by local recurrent connections in layer-5. Previous studies have shown a relatively high recurrence and numerous other analogous characteristics of the layer-5 pyramidal cells and LSTMs. \cite{douglas_koch_mahowald_martin_suarez_1995, thomson_2002, kerkoerle_self_roelfsema_2017, goldman-rakic_1995} The slow memory decay in layer-5 networks may be controlled by short-term and long-term plasticity at recurrent excitatory synapses. \cite{abbott_nelson_2000} The gates that are used by LSTMs are thought to be implemented by the lateral inhibitory inputs. The \textit{input gate} is proposed to be implemented by inhibitory neurons in layer-2/3 or layer-4.

\subsection{Spiking Neural Networks}
The current basis for computing biological neuron activity is the action potential. When a presenaptic cell fires it releases a neurotransmitter via its synaptic terminals. It then causes the flow of electrically charged ions into or out of the postsynaptic cell. It may increase (depolarise) or decrease (hyperpolarise) the postsynaptic membrane potential. Sufficient depolarisation will cause the postsynaptic potential to reach a threshold and fire a spike (action potential, \cite{kandel_schwartz_jessell_2000}). This behaviour can be formalised as an RC-circuit which provieds the leaky-integrate-and-fire neuron model (LIF). The LIF model gives an approximation of the firing rates \cite{gerstner_kistler_2008} given in equation (\ref{eq:LIF_rate}), where $\tau_m = RC$ is the time constant given by the RC-circuit, $I_{exc}$ is the excitatory input current and $I_{inh}$ is the inhibitory input current.
\begin{equation}\label{eq:LIF_rate}
    \text{rate} \sim  \left(\tau_m \ln\frac{R(I_{exc} - I_{inh})}{R(I_exc - I_inh) - 
    \theta} \right)^{-1}
\end{equation}
This rate approximation in equation (\ref{eq:LIF_rate}) shows that the inhibition (gating) is subtractive rather than multiplicative like in canonical LSTMs. This provides the basis for introducing a more biologically plausible recurrent model.
\section{Methods}
The biological inspiration gives foundations for proposing a new architecture, which uses subtractive gating rather than multiplicative. The LSTM implementation using subtractive gating rather than multiplicative is called \textbf{subLSTM}. The architecture and the comparisons have been displayed on \figurename{\ref{fig:subLSTM}}.

\columnFigure{subLSTM}{\textbf{a.} Example unit of a simplified cortical recurrent neural network. Sensory input projects to pyramidal cells (PC) in layer-2/3, which then propagates to the memory cell. The memory decays with a decay time constant $f$. The input onto layer-5 is balanced out by inhibitory basket cells (BC). The balance is represented by the diagonal 'equal' connection. The graph below shows the gating states for an implementation using a simple leaky-integrate-and-fire model capped to 200 Hz with subtractive inhibition. \textbf{b.} The implementation of the cortical model following the similar notation to LSTM units, but with $\textbf{i}_t$ and $\textbf{o}_t$ being subtractive input and output gates. Dashed connections represent the potential to have a balance between excitatory and inhibitory input. The graph below shows sigmoidal activation functions with subtractive gating. \textbf{c.} An LSTM recurrent neural network cell. The graph below shows the sigmoidal activation functions using multiplicative gating. The bottom graphs represent the frequency of spikes in Hz as in biological circuits.}

A LSTM unit consists of a memory cell $\textbf{c}_t$, input $\textbf{i}_t$, forget $\textbf{f}_t$ and output $\textbf{o}_t$ gates. The state of the LSTM is defined as $\textbf{h}_t = f(\textbf{x}_t, \textbf{h}_{t-1}, \textbf{i}_t, \textbf{f}_t, \textbf{o}_t)$, where $\textbf{x}_t$ is the input vector. The dynamics of the unit are given below.
\begin{table}[h!]
\centering
\def\arraystretch{1.5}
\begin{tabular}{c|c|c}
        & \textbf{LSTM}               & \textbf{subLSTM}             \\
$[\textbf{f}_t, \textbf{o}_t, \textbf{i}_t]^T$ = & $\sigma(W\textbf{x}_t + R\textbf{h}_{t-1} + \textbf{b}$ & $\sigma(W\textbf{x}_t + R\textbf{h}_{t-1} + \textbf{b}$\\
$\textbf{z}_t$ = & $\tanh(W\textbf{x}_t + R\textbf{h}_{t-1} + \textbf{b}$ & $\sigma(W\textbf{x}_t + R\textbf{h}_{t-1} + \textbf{b}$\\
$\textbf{c}_t$ = & $\textbf{c}_{t-1} \odot \textbf{f}_t + \textbf{z}_t \odot i_t$ & $\textbf{c}_{t-1} \odot \textbf{f}_t + \textbf{z}_i - \textbf{i}_t$  \\
$\textbf{h}_t$ = & $\tanh(\textbf{c}_t) \odot \textbf{o}_t$             & $\sigma(\textbf{c}_t) - \textbf{o}_t$          
\end{tabular}
\end{table}
Here, $\odot$ denotes element-wise multiplication, $W$, $R$ and $\textbf{b}$ are the parameters of the models and $\sigma(\cdot)$ is the sigmoid function. The major difference occurs in $\textbf{c}_t$ and $\textbf{h}_t$, where the multiplication is changed to subtraction.


\subsection{Gating Solutions}
Canonical LSTM models contain an input gate \textit{i}, an output gate \textit{o} and a forget gate \textit{f}. Input gate and output gate have been discussed previously and stated that the subtractive inhibitory connections exhibit gating characteristics. The forget gate has two possible implementations:
\begin{enumerate}
    \item using a forget fate similar to LSTMs.
    \item using a learned simple decay, which is more biologically plausible.
\end{enumerate}
The latter method called \textbf{fix-subLSTM} was also examined.

\subsection{Gradient comparison}
The authors of the paper have performed a gradient comparison and found that subLSTMs don't directly impair the error propagation. Ergo subtractive gating has the potential to improve gradient flow towards the input layers.
\section{Results}
The three models (subLSTM, fix-subLSTM and LSTM) were compared empirically on two tasks (MNIST \cite{lecun-mnisthandwrittendigit-2010} and world-level language modelling \cite{marcus_1993, DBLP:journals/corr/MerityXBS16}). The network weights were initialised with Glorot initialisation \cite{Glorot10understandingthe} and the LSTM units had an initial forget bias equal to 1. The number of parameters was held constant across models for better comparison.

\subsection{Sequential MNIST}
The "sequential" MNIST is a digit classification taks, where each digit image is decomposed from $28 \times 28$ pixels to 784 steps (\figurename{\ref{fig:mnist}a}). The networks were opitmised using RMSProp with momentum, a learning rate of $10^{-4}$, one hidden layer and 100 hidden units. The results show that subLSTMs achieve similar results to LSTMs (\figurename{\ref{fig:mnist}b}). The results were comparable to previous results using the same task \cite{DBLP:journals/corr/LeJH15}.
\columnFigure{mnist}{A comparison of LSTM and subLSTM networks for sequential MNIST using 100 hidden units. \textbf{a.} Examples from the MNIST dataset. Each matrix was converted into a sequence of 784 elements. \textbf{b.} Classification accuracy on the test set.}

\subsection{Language Modelling}
Language modelling is more challenging task for RNNs as it requires both short and long-term dependencies. The task was to predict the following word at each timestep. For evaluation \textit{preplexity} was used which measures how well a probability model predicts a sample. Each model had 2 hidden layers and backpropagation was truncated to 35 steps with a batch size of 20. RMSProp with momentum was used. A hyperparameter search was performed using Google Vizier
The test was performed on a number of hidden units: 10, 100, 200, 650. The number of parameters was kept constant across models. Training was done on Penn Treebank and Wikitext-2.
\begin{table}[h!]
\caption{Penn Treebank (PTB) Test Perplexity}\label{tab:penn}
\centering
\begin{tabular}{llll}
\hline
size & subLSTM & fix-subLSTM & LSTM   \\ \hline
10   & 222.80  & \textbf{213.86}      & 215.93 \\
100  & 91.46   & 91.84       & \textbf{88.39}  \\
200  & 79.59   & 81.97       & \textbf{74.60}  \\
650  & 76.17   & 70.58       & \textbf{64.34}  \\ \hline
\end{tabular}
\end{table}
\begin{table}[h!]
\caption{Wikitext-2 Test Perplexity}\label{tab:wiki}
\centering
\begin{tabular}{llll}
\hline
size & subLSTM & fix-subLSTM & LSTM   \\ \hline
10   & 268.33  & \textbf{259.89}      & 271.44 \\
100  & 103.36   & 105.06       & \textbf{102.77}  \\
200  & 89.00   & 94.33       & \textbf{86.15}  \\
650  & 78.92   & 79.49       & \textbf{74.27}  \\ \hline
\end{tabular}
\end{table}
Tables \ref{tab:penn} and \ref{tab:wiki} show that on both datasets the models perfomed similarily, with the fix-subLSTM sometimes performing better than the subLSTM.
\section{Conclusion}
%
Cortical microcircuits have complex network architectures which support extensive dynamics, however their computational power is yet to be understood. Making sense of excitatory and inhibitory interactions is outside the scope of current experimental approaches, albeit being known that they interact closely in the porcessing of sensory information. It has been shown that biologically constrained LSTMs could perform similarly to generic LSTMs and these subtractively gated excitation-inhibition recurrent neural networks show promise for the future research. It is expected that some biological tricks of the cortical networks may boost LSTMs performance in the future.
\section{Discussion}
It is important to note that subLSTMs do not outperform traditional LSTMs, however they show that biology might be a good source of information for the future.
On the other hand the proposed model subLSTM and cortical model are in general a very vague abstraction of the real cortical processing, concentrating on solely one aspect of the biological network, which is the subtractive gating. There is much more complexity to the system than it is presented, as the Dale's principle (a given neuron can only be inhibitory or excitatory) was omitted as well as the shunting inhibition. The authors of the work however mention those as the next step in the their work is attempting to include more biological principles.
What is more all the models and results were implemented by the authors, leaving to wonder if the state-of-the-art implementations wouldn't have drawn a less positive picture for biologically constrained implementations.

%
\bibliographystyle{IEEEtran}
\bibliography{references}


\end{document}
