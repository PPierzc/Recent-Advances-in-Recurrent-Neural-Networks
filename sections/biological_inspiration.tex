\section{Biological Inspiration}
The work was inspired by the columnar architecture of the cortical circuits presented on \figurename{\ref{fig:biomodel}}. It partially describes the flow of sensory information from the thalamus to the neocortex. The sensory input is propagated to layer 2/3 pyramidal cells. The output of layer 2/3 is projected to layer-5 pyramidal cells, where recurrent units may form. The input into the layer-5 pyramidal neurons is gated by local inhibitory connections. As a result it may retain a given input for some time. The timescale of this is dependant on the decay time constant. The output of layer-5 is then projected to other units of the network, with the output controlled by another inhibitory neuron.

\columnFigure{biomodel}{Biological architecture of the neocortex, which resembles LSTM features. \textbf{1.} The thalamus projects the sensory input to layer 2/3 pyramidal cells. \textbf{2.} The output of layer 2/3 pyramidal cells is projected to layer-5 pyramidal cells. These may form recurrent units analogous to the memory cell of LSTMs. Here local interneurons control gate the input into the memory cell (LSTM input gates \textit{i}). \textbf{3.} The recurrences allow for retaining a given input for some time dependant on the decay time constant \textit{f}. \textbf{4.} Finally the output is sent to other units, where the output is gated by another local inhibitory cell (LSTM output gate \textit{o}).}

\subsection{LSTMs vs. Biological Circuits}
When compared to LSTMs significant resemblance can be found. The central element of LSTMs, the \textit{memory cell} is hypothesised to be implemented by local recurrent connections in layer-5. Previous studies have shown a relatively high recurrence and numerous other analogous characteristics of the layer-5 pyramidal cells and LSTMs. \cite{douglas_koch_mahowald_martin_suarez_1995, thomson_2002, kerkoerle_self_roelfsema_2017, goldman-rakic_1995} The slow memory decay in layer-5 networks may be controlled by short-term and long-term plasticity at recurrent excitatory synapses. \cite{abbott_nelson_2000} The gates that are used by LSTMs are thought to be implemented by the lateral inhibitory inputs. The \textit{input gate} is proposed to be implemented by inhibitory neurons in layer-2/3 or layer-4.

\subsection{Spiking Neural Networks}
The current basis for computing biological neuron activity is the action potential. When a presenaptic cell fires it releases a neurotransmitter via its synaptic terminals. It then causes the flow of electrically charged ions into or out of the postsynaptic cell. It may increase (depolarise) or decrease (hyperpolarise) the postsynaptic membrane potential. Sufficient depolarisation will cause the postsynaptic potential to reach a threshold and fire a spike (action potential, \cite{kandel_schwartz_jessell_2000}). This behaviour can be formalised as an RC-circuit which provieds the leaky-integrate-and-fire neuron model (LIF). The LIF model gives an approximation of the firing rates \cite{gerstner_kistler_2008} given in equation (\ref{eq:LIF_rate}), where $\tau_m = RC$ is the time constant given by the RC-circuit, $I_{exc}$ is the excitatory input current and $I_{inh}$ is the inhibitory input current.
\begin{equation}\label{eq:LIF_rate}
    \text{rate} \sim  \left(\tau_m \ln\frac{R(I_{exc} - I_{inh})}{R(I_exc - I_inh) - 
    \theta} \right)^{-1}
\end{equation}
This rate approximation in equation (\ref{eq:LIF_rate}) shows that the inhibition (gating) is subtractive rather than multiplicative like in canonical LSTMs. This provides the basis for introducing a more biologically plausible recurrent model.