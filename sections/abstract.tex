\begin{abstract}
A summary paper of "\textit{Cortical microcircuits as gated-recurrent neural networks}" by Rui Ponte Costa et al. Cortical circuits show recurrent architectures across different brain areas. This suggest existance of computational principles, which still remain elusive. The authors propose a modified method of long short-term memory networks, which are meant to be more biologically plausible. They introduce a network model, which uses inhibitory cells as subtractive gates (subLSTMs). This allowed them to propose a mapping of subLSTMs onto known cortical excitatory-inhibitory cortical microcircuits. Empirical evaluation was performed on sequential image classification and language modelling. The results suggest that cortical circuits could be optimised to solve complex contextual problems. Their work provides a step towaeds unifying deep learning with their biological counterparts.
\end{abstract}