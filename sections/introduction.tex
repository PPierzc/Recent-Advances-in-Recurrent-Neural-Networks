\section{Introduction}
This is a summary of "\textit{Cortical microcircuits as gated-recurrent neural networks}" by Rui Ponte Costa et al \cite{PonteCosta2017}.
The past decades have unveiled complex but stereotypical structures within the neocortex. The most prevalent feature of the cortical networks, being their laminar organisation and a high degree of recurrence \cite{douglas_koch_mahowald_martin_suarez_1995, jiang_shen_cadwell_berens_sinz_ecker_patel_tolias_2015}. Thus Recurrent Neural Networks (RNNs) and Long-Short Term Memory units (LSTMs) seem like an appealing idea for modelling cortical structures. There is however a major difference between the two architectures. LSTMs have a multiplicative factor for gating while biological networks use inhibitory neurons, which have a subtractive nature. There have been suggestions of LSTMs as models of working memory and different brain areas \cite{krueger_dayan_2009}, but without clear interpretation of the individual components of LSTMs and specific mapping to known circuits.
By transferring the functionality of LSTMs into a more biologically plausible network the work is meant to provide a testable hypotheses for experiments on the functionality of entire cortical microcircuits.